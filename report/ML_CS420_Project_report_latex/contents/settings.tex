\section{Experimental settings}

\subsection{训练参数}

网络超参数为第 \ref{sec:methods} 节所言的默认参数。训练时参数如表 \ref{tab:param} 所示。

\begin{table}[ht]
    \centering
    \caption{训练参数}
    \label{tab:param}
    \begin{tabular}{cccccc}
        \toprule
        参数 & Sketch-a-Net &  AlexNet & ResNet & BiLSTM & CNN+RNN \\
        \midrule
        lr &  \multicolumn{5}{c}{0.001} \\
        batch size & \multicolumn{5}{c}{64}  \\
        weight decay & 0.001 &  0.001&  0.001& 0.0006  &0.0006 \\
        % \midrule
        % RNN 隐藏层元胞数 & - & - & - & $256\times 2$ & $256\times 2$ \\
        \bottomrule
    \end{tabular}
\end{table}

\subsection{优化训练方法}

\subsubsection{权重衰减}
    权重衰减(weight decay),也可称为L2正则化,其目的是让权重衰减到更小的值,在一定程度上缓解模型过拟合的问题。其之所以有效,是因为L2范数对于权重向量的大分量施加了巨大的惩罚,使学习算法会偏向在大量特征上均匀分布权重的模型,对单个变量上的观测误差更加稳定,而非将权重集中在小部分特征上。
    
    本小组通过设置Adam优化器中 \verb"weight_decay",以抑制模型深度可能过深带来的过拟合现象,增强模型的泛化能力。
\subsubsection{可变的学习率}
    调整学习率也是优化模型十分重要的一环。如果学习率过大,会造成结果难以收敛;而如果学习率太小,既可能导致结果收敛过慢,训练时间过长,也可能导致结果只处于一个局部最优而非更好的结果。因此,在训练的过程当中,我们需要动态衰减学习率。
    
    我们使用了 PyTorch 包中自带的scheduler类来进行学习率衰减,如下面代码所示:
    \begin{lstlisting}
     scheduler = ReduceLROnPlateau(optimizer, 'min', factor=0.3, patience=2)
     train_model(model, dataloaders_dict, criterion, optimizer, scheduler, num_epochs=5)
    \end{lstlisting}
    解释:
    \begin{itemize}
        \item \textbf{patience}:这个参数是容忍程度的意思,意味着在训练数量为patience的轮数之后,如果validation集上的loss一直则没有下降,则要求开始降低学习率。本小组的代码中将这设置成2。
        \item \textbf{factor}:这个参数是学习率衰减时的乘以的因子,本小组将此设置成0.3,意味着每一次衰减为原来的0.3倍。
    \end{itemize}

